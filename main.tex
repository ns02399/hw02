%CS-113 S18 HW-2
%Released: 2-Feb-2018
%Deadline: 16-Feb-2018 7.00 pm
%Authors: Abdullah Zafar, Emad bin Abid, Moonis Rashid, Abdul Rafay Mehboob, Waqar Saleem.


\documentclass[addpoints]{exam}

% Header and footer.
\pagestyle{headandfoot}
\runningheadrule
\runningfootrule
\runningheader{CS 113 Discrete Mathematics}{Homework II}{Spring 2018}
\runningfooter{}{Page \thepage\ of \numpages}{}
\firstpageheader{}{}{}

\boxedpoints
\printanswers
\usepackage[table]{xcolor}
\usepackage{amsfonts,graphicx,amsmath,hyperref}
\title{Habib University\\CS-113 Discrete Mathematics\\Spring 2018\\HW 2}
\author{ns02399}  % replace with your ID, e.g. oy02945
\date{Due: 19h, 16th February, 2018}


\begin{document}
\maketitle

\begin{questions}



\question

%Short Questions (25)

\begin{parts}

 
  \part[5] Determine the domain, codomain and set of values for the following function to be 
  \begin{subparts}
  \subpart Partial
  \subpart Total
  \end{subparts}

  \begin{center}
    $y=\sqrt{x}$
  \end{center}

  \begin{solution}
  
    A partial function isn't defined over some values of the domain so the function $y = \sqrt{x}$ is partial over the domain; $x \in \mathbb{Z}$ and over the co-domain $y \in \mathbb{R}$(positive), since for e.g x = -3 is not mapped to any value in the co-domain.
    Where as a total function is defined for all domain values so the function $y = \sqrt{x}$ is total for the domain $x \in \mathbb{N}$ and co-domain $y \in \mathbb{R}$(positive), since all natural numbers are mapped to the co-domain.
    
  \end{solution}
  
  \part[5] Explain whether $f$ is a function from the set of all bit strings to the set of integers if $f(S)$ is the smallest $i \in \mathbb{Z}$� such that the $i$th bit of S is 1 and $f(S) = 0$ when S is the empty string. 
  
  \begin{solution}
    
    A function is one that maps all the values of the domain to at least one value of the co-domain. Since $f$ is undefined for a string consisting of all zeroes.
    
  \end{solution}

  \part[15] For $X,Y \in S$, explain why (or why not) the following define an equivalence relation on $S$:
  \begin{subparts}
    \subpart ``$X$ and $Y$ have been in class together"
    \subpart ``$X$ and $Y$ rhyme"
    \subpart ``$X$ is a subset of $Y$"
  \end{subparts}

  \begin{solution}
    
    For a set to have an equivalence relation, it must also be reflexive, symmetric, and transitive.\\ \\
    i. If A is in class with B and B is in a different class with C then A and B are not in the same class together. Transitive property is not met so this is not an equivalence relation.\\ \\
    ii. This is an equivalence relation because it is reflexive as every word rhymes with itself and symmetric because if A rhymes with B then B also rhymes with A and transitive because if A rhymes with B and B rhymes with C then A must also rhyme with C.\\ \\
    iii. Since the symmetric property isn't met, this isn't an equivalence relation as it's not necessary for B to be a subset of A if A is a subset of B. consider the sets;
    A = {1,2,3}
    B = {1,2,3,4}
    
  \end{solution}

\end{parts}

%Long questions (75)
\question[15] Let $A = f^{-1}(B)$. Prove that $f(A) \subseteq B$.
  \begin{solution}
    
    Taking inverse on both sides of the equation gives us $f(A) = B$. Because of the law of equality if A is a subset and equal to B then B should also be a subset of and equal to A. Considering a, b, and c to be elements of A that map to elements of B; 10, 11, and 12 through $f$ respectively. It is observable that $f(A) = 10, 11, 12$ is a subset of B and also equal to B.
    
  \end{solution}

\question[15] Consider $[n] = \{1,2,3,...,n\}$ where $n \in \mathbb{N}$. Let $A$ be the set of subsets of $[n]$ that have even size, and let $B$ be the set of subsets of $[n]$ that have odd size. Establish a bijection from $A$ to $B$, thereby proving $|A| = |B|$. (Such a bijection is suggested below for $n = 3$) 

\begin{center}

  \begin{tabular}{ |c || c | c | c |c |}
    \hline
 A & $\emptyset$ & $\{2,3\}$ & $\{1,3\}$ & $\{1,2\}$ \\ \hline
 B & $\{3\}$ & $\{2\}$ & $\{1\}$ & $\{1,2,3\}$\\\hline
\end{tabular}
\end{center}

  \begin{solution}
    
Defining $f$ by the rule $f(A) = x - \{1\} $ if $\{1\} \in A$ or $f(A) = x  \cup \{1\}$ if $\{1\} \notin A$.\\
Assuming $A_1$ and $A_2$ to be two subsets of A. If \{1\} \in $A_1$, then \{1\} \notin $f(A_1)$.\\
Consecutively \{1\} \notin  $f(A_2)$ so \{1\} \in $A_2$.\\
Concluding that $A_1$ = $A_2$, means that $f$ is injective. Similarly if $\{1\} \in B$ then $B - \{1\}$ should map to B by the defined function. Similarly if $\{1\} \notin B$ then $B \cup \{1\}$ should map to B, Concluding that all image elements are mapped to the the domain elements makes the function surjective.
  \end{solution}
  
\question Mushrooms play a vital role in the biosphere of our planet. They also have recreational uses, such as in understanding the mathematical series below. A mushroom number, $M_n$, is a figurate number that can be represented in the form of a mushroom shaped grid of points, such that the number of points is the mushroom number. A mushroom consists of a stem and cap, while its height is the combined height of the two parts. Here is $M_5=23$:

\begin{figure}[h]
  \centering
  \includegraphics[scale=1.0]{m5_figurate.png}
  \caption{Representation of $M_5$ mushroom}
  \label{fig:mushroom_anatomy}
\end{figure}

We can draw the mushroom that represents $M_{n+1}$ recursively, for $n \geq 1$:
\[ 
    M_{n+1}=
    \begin{cases} 
      f(\textrm{Cap\_width}(M_n) + 1, \textrm{Stem\_height}(M_n) + 1, \textrm{Cap\_height}(M_n))  & n \textrm{ is even} \\
      f(\textrm{Cap\_width}(M_n) + 1, \textrm{Stem\_height}(M_n) + 1, \textrm{Cap\_height}(M_n)+1) & n \textrm{ is odd}  \\      
   \end{cases}
\]

Study the first five mushrooms carefully and make sure you can draw subsequent ones using the recurrence above.

\begin{figure}[h]
  \centering
  \includegraphics{mushroom_series.png}
  \caption{Representation of $M_1,M_2,M_3,M_4,M_5$ mushrooms}
  \label{fig:mushroom_anatomy}
\end{figure}

  \begin{parts}
    \part[15] Derive a closed-form for $M_n$ in terms of $n$.
  \begin{solution}
    
    The Row of dots directly above the stem can be represented as always having $n + 1$ dots.\\ \\
    The Stem itself can be represented as always having 2(n-1) dots in 2 columns.\\ \\
    The rest of the cap (excluding the base of the cap is a bit tricky. The number of rows (X) in the cap but above the cap base can be represented as $X = [n//2]$.\\
    Because each row contains 1 dot less than the dots in the row below itself, these can be represented as $X(n + 1) - X((X+1)/2)$ dots.\\ \\
    To calculate total dots we just have to sum the three expressions which will give us:\\
    $2(n-1) + (n+1) + (n+1)(X) - X((X+1)/2)$.\\
    $M_n$ comes out be:\\
    $(n+1)(n//2 + 1) - [(n//2)(n//2) + 1]/2 + 2(n-1)$
    
  \end{solution}
    \part[5] What is the total height of the $20$th mushroom in the series? 
  \begin{solution}
    
    Stem = $n - 1$.\\
    Cap = Height of Remaining Cap $+ 1$ = $(n//2) + 1$.\\
    Total Height = Stem + Cap\\
    = $n + (n/2)$.\\ \\
    So the height of 20th mushroom = $20 + 10 = 30$
    
  \end{solution}
\end{parts}

\question
    The \href{https://en.wikipedia.org/wiki/Fibonacci_number}{Fibonacci series} is an infinite sequence of integers, starting with $1$ and $2$ and defined recursively after that, for the $n$th term in the array, as $F(n) = F(n-1) + F(n-2)$. In this problem, we will count an interesting set derived from the Fibonacci recurrence.
    
The \href{http://www.maths.surrey.ac.uk/hosted-sites/R.Knott/Fibonacci/fibGen.html#section6.2}{Wythoff array} is an infinite 2D-array of integers where the $n$th row is formed from the Fibonnaci recurrence using starting numbers $n$ and $\left \lfloor{\phi\cdot (n+1)}\right \rfloor$ where $n \in \mathbb{N}$ and $\phi$ is the \href{https://en.wikipedia.org/wiki/Golden_ratio}{golden ratio} $1.618$ (3 sf).

\begin{center}
\begin{tabular}{c c c c c c c c}
 \cellcolor{blue!25}1 & 2 & 3 & 5 & 8 & 13 & 21 & $\cdots$\\
 4 & \cellcolor{blue!25}7 & 11 & 18 & 29 & 47 & 76 & $\cdots$\\
 6 & 10 & \cellcolor{blue!25}16 & 26 & 42 & 68 & 110 & $\cdots$\\
 9 & 15 & 24 & \cellcolor{blue!25}39 & 63 & 102 & 165 & $\cdots$ \\
 12 & 20 & 32 & 52 & \cellcolor{blue!25}84 & 136 & 220 & $\cdots$ \\
 14 & 23 & 37 & 60 & 97 & \cellcolor{blue!25}157 & 254 & $\cdots$\\
 17 & 28 & 45 & 73 & 118 & 191 & \cellcolor{blue!25}309 & $\cdots$\\
 $\vdots$ & $\vdots$ & $\vdots$ & $\vdots$ & $\vdots$ & $\vdots$ & $\vdots$ & \color{blue}$\ddots$\\
 

\end{tabular}
\end{center}

\begin{parts}
  \part[10] To begin, prove that the Fibonacci series is countable.
 
    \begin{solution}
    
    A countable set is one that has a one to one correspondence with natural numbers \mathbb{N}.\\
    So any set that has a bijection with the set of Natural Numbers is called countably infinite.\\
    Because the fibonacci sequence can be mapped to natural numbers bijectively. So we can say that the fibonacci series is countable.
    
  \end{solution}
  \part[15] Consider the Modified Wythoff as any array derived from the original, where each entry of the leading diagonal (marked in blue) of the original 2D-Array is replaced with an integer that does not occur in that row. Prove that the Modified Wythoff Array is countable. 

  \begin{solution}
    % Write your solution here
  \end{solution}
\end{parts}

\end{questions}

\end{document}
